%!TEX root = ../../report.tex
\chapter*{Abstract}
This paper describes the design and implementation of a ping pong ball catcher.
The purpose of the system is to catch a ball, dropped onto a platform, after it has bounced once.
The point of impact of a ball can be determined based on time-difference of arrival of the waves expanding through the platform material from the impact point.
A platform consisting of a square base with four piezoelectric sensors in the corners, and an actuated arm with a net attached, is built.
Analogue circuits are designed to pre-process sensor signals to interface with an FPGA. A component to precisely measure time differences is written in VHDL.
Fast calculation of the impact point in implemented in C on a soft microprocessor.
A desktop GUI application is programmed to obtain and visualise data for each ball hit.
The measurements of time differences turn out to be too inconsistent to always catch the ball.
