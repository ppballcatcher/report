%!TEX root = ../../report.tex
\chapter{Introduction}
\label{chap:introduction}
%	\section{Overall description}
		The goal of this project is catching a ping pong ball released above a platform, on which four piezoelectric elements are placed in a rectangular pattern.
		When the ball hits the platform once, each of the piezoelectric elements generates a signal and permit multilateration of the impact point. When the impact point has been found, a net will catch the ball when it returns toward the platform after the first bounce.

		This project has background in similar projects as described in \cite{electronic_target} where systems resembling this is used to determine a hit position of a bullet. Also in \cite{tdoa_book} and \cite{tdoa_notes} this method is described and proved. This background motives the project because it shows that the idea is possible and reachable.
		%
	\section{Report structure}
	\label{sec:reportStructure}
		The report is structured as follows. To start, the overall design principles \ref{chap:overall_design_principles} where the first theoretical approach to the problem is presented.
		Then, the mechanical view of the project is presented, analyzed and solved \ref{chap:mechanics}.
		Followed by the electronics part where the same point of view is done \ref{chap:electronics}.
		Continues with the programming part which is separated into a low-level and a high-level part.
		In the low-level part \ref{chap:low_level_programming} the VHDL and MicroBlaze sections are treated while in the high-level programming \ref{chap:high_level_programming} the explanation of the computer program is.
		To conclude, the report finish with the experiments carried out \ref{chap:experiments}, the discussion \ref{chap:discussion}, where future works are commented, and the conclusions \ref{chap:conclusion}.
		At the end is the appendix which contains the mechanical drawings \ref{chap:mechanical_drawings}, the Digi diagrams \ref{chap:digi_diagrams} and the equipment used for the project \ref{chap:equipment}.