\chapter{Introduction}
\label{chap:introduction}

	\section{Overall description}
		The goal of this project is catching a ping pong ball released above a platform, on which four piezoelectric elements are placed in a rectangular pattern. 
		Once the ball has hit the platform once, each of the piezo elements will generate a signal and permit triangulation of the impact point. When the impact point has been found, a net will catch the ball when it returns toward the platform after the first bounce.

		This project has background similar projects as described in \cite{electronic_target} where kind of this system is used for find out the hit's position of a bullet. Also in \cite{tdoa_book} and \cite{tdoa_notes} this method is described and proved. This background motives more the project because it shows that the idea it is possible and reachable.

	\section{Report structure}
	\label{sec:reportStructure}
		The report is structured as follows. To start, the overall design principles \ref{chap:overall_design_principles} where the first theoretical approach to the problem is presented. 
		Then, the mechanical view of the project is presented, analyzed and solved \ref{chap:mechanics}. 
		Follows with the electronics part where the same point of view is done \ref{chap:electronics}. 
		Continues with the programming part where two parts are differentiated, the low-level programming and the high-level. 
		In the first one \ref{chap:low_level_programming} the VHDL and Microblaze sections are treated while in the high-level programming \ref{chap:high_level_programming} the explanation of the computer program is. 
		To conclude, the report finish with the experiments carried out \ref{chap:experiments}, the discussion \ref{chap:discussion}, where future lines are commented, and the conclusions \ref{chap:conclusions}.
		At the end can be found appendix as the mechanical drawings \ref{chap:mechanical_drawings}, the Digi diagrams \ref{chap:digi_diagrams} and the equipment used for the project \ref{chap:equipment}.