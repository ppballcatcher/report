%!TEX root = ../../report.tex
\chapter{Conclusion}
\label{chap:conclusion}
	A ping pong ball catcher has been implemented and tested for its ability to catch balls after one bounce on a platform.

	A tower with a servo motor mounted have been constructed. It enables the servo to actuate a net on an arm to catch bouncing ping pong balls.
	The balls hits a metal plate with piezo sensors mounted on.
	Four PCB boards have been constructed that enables conversion of alternating bending waves in the platform to digital values compatible with digital pins of the used FPGA.
	A cross platform GUI PC-program have been develop which have proven very intuitive to use for the authors of this report.

	The program are able to show calculated points of impact of balls.
	Regarding the low level programming a VHDL project have been implemented to measure time difference between detected digitalized wave fronts. This operation have been verified in simulation and practice to be correct.
	A solution to the TDOA problem have been derived and implemented on a MicroBlaze using Gauss-elimination.
	It has been verified with MATLAB on multiple real measurements.

	The designed machine are able catch a ping-pong ball on its way down after a bounce. when the position are calculated correctly.
	The positioning system have however proven very unstable which is due to variations in the percepted waves.

	Possible improvements regarding the determination of the TDOA and the physical setup have been located, but not implemented.

%	On the other hand, we haven't develop a system able to catch all the balls though the problems that cause this has been located and possible solutions has been presented.

%	Also we believe that we have not only shown that we have  achieved the requirements of the course but we have further develop our knowledge in fields like FEM, MicroBlaze or digital design.

	%	After designing and implementing a
	%	After the design and implementation of the project and have carried out all the experiments to determine the reach of the machines abilities. It is concluded that we have achieved the general purpose of the project, but not the reliability and the precision we hoped for.
	%	On one hand, the project has been split in a way that has let us work in parallel and in an efficient way so each person in the group has had workload in every moment.
